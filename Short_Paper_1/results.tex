\documentclass[11pt]{article}
%\documentclass[11pt]{report}
\usepackage{amsmath}
\usepackage{amsfonts}
\usepackage{amssymb}
\usepackage{graphicx,subfigure} %perch� mi d� prob nella compilazione
\usepackage{booktabs}
\usepackage{color}
\usepackage[applemac]{inputenc}
\usepackage[english]{babel}
\usepackage{cancel}
\usepackage{dsfont}
\usepackage{hyperref}

\def\red{\color{red}}
\def\black{\color{black}}
\def\blue{\color{blue}}
\def\green{\color{green}}
\def\gray{\color{gray}}
\def\cyan{\color{cyan}}
\def\magenta{\color{magenta}}
\def\yellow{\color{yellow}}
\def\violet{\color{violet}}
\def\yellow{\color{yellow}}

%\def\blue{\color{blue!80!cyan}}

% \def\blue{\color{blue}}
 \def\cre{\color{red}}
 \def\cbl{\color{black}}
\definecolor{eublue}{rgb}{0.1,0.1,0.5} % Definition of blue color
\definecolor{eumagenta}{rgb}{1,0,0.9} % Definition of blue color


\setlength{\textwidth}{16.5cm} \setlength{\textheight}{24cm}
\setlength{\oddsidemargin}{-0cm}
\setlength{\evensidemargin}{-0.5cm} \setlength{\topmargin}{-2cm}
\renewcommand{\baselinestretch}{1.3}
\newcommand{\ud}{\mathrm{d}}


%%% stuff for theorem-like environments
\newtheorem{definition}{De�finition}
\newtheorem{theorem}{The�ore�m}
\newtheorem{lemma}{Lemme}
\newtheorem{remark}{Remarque}
\newtheorem{proposition}{Proposition}
\newtheorem{corollary}{Corollaire}
\newcommand{\bpf}{{\bf Preuve }}
\newcommand{\epf}{\begin{flushright}$\blacksquare$\end{flushright}}


\newcommand{\BE}{\begin{equation}}
\newcommand{\BEno}{\begin{equation*}}
\newcommand{\EE}{\end{equation}}
\newcommand{\EEno}{\end{equation*}}
\newcommand{\mtx}[1]{\left[\begin{matrix}#1\end{matrix}\right]}

%%% probability symbolism
%\newcommand{\operatorname}{\mathop}
\newcommand{\logit}[1]{\operatorname{logit}\left(#1\right)}
\newcommand{\prob}[1]{\operatorname{\mathbb P}(#1)}
\newcommand{\probstar}[1]{\operatorname{P}^*(#1)}
\newcommand{\expect}[1]{\operatorname{\mathbb E}\left[#1\right]}
\newcommand{\expectsub}[2]{\expectation_{#1}\left(#2\right)}
\newcommand{\variance}[1]{\operatorname{Var}\left(#1\right)}
\newcommand{\covariance}[2]{\operatorname{Cov}\left(#1,#2\right)}
\newcommand{\correlation}[2]{\operatorname{Corr}\left(#1,#2\right)}
\newcommand{\normaldensity}[3] {\frac1{\sqrt{2\pi#3}}\exp\left[-\frac1{2#3} (#1-#2)^2\right]}

\newcommand{\trace}[1]{\operatorname{trace}\left(#1\right)}

\newcommand{\iid}{\stackrel{\mathrm{iid}}{\sim}}
\newcommand{\app}{\stackrel{\mathrm{app}}{=}}
%%% different fields in mathematics
\newcommand{\naturals}{\mathbb{N}}
\newcommand{\reals}{\mathbb{R}}
\newcommand{\posreals}{\reals_+}
\newcommand{\realvectors}[1]{\reals^{#1}}
\newcommand{\complex}{\mathbb{C}}
\newcommand{\integers}{\mathbb{Z}}

\newcommand{\mT}{\mathcal{T}}
\newcommand{\mF}{\mathcal{F}}
\newcommand{\mS}{\mathcal{S}}
\newcommand{\mN}{\mathcal{N}}
\newcommand{\mP}{\mathcal{P}}
\newcommand{\mC}{\mathcal{C}}
\newcommand{\mU}{\mathcal{U}}
\newcommand{\mR}{\mathcal{R}}

\newcommand{\bzero}{\text{\mathversion{bold}$0$\mathversion{normal}}}
\newcommand{\buno}{\text{\mathversion{bold}$1$\mathversion{normal}}}
%%% and greek
\newcommand{\bSigma}{\text{\mathversion{bold}$\Sigma$\mathversion{normal}}}
\newcommand{\bmu}{\text{\mathversion{bold}$\mu$\mathversion{normal}}}
\newcommand{\bnu}{\text{\mathversion{bold}$\nu$\mathversion{normal}}}
\newcommand{\btheta}{\text{\mathversion{bold}$\theta$\mathversion{normal}}}
\newcommand{\brho}{\text{\mathversion{bold}$\rho$\mathversion{normal}}}
\newcommand{\balpha}{\text{\mathversion{bold}$\alpha$\mathversion{normal}}}
\newcommand{\bbeta}{\text{\mathversion{bold}$\beta$\mathversion{normal}}}
\newcommand{\bGamma}{\text{\mathversion{bold}$\Gamma$\mathversion{normal}}}
\newcommand{\bLambda}{\text{\mathversion{bold}$\Lambda$\mathversion{normal}}}
\newcommand{\blambda}{\text{\mathversion{bold}$\lambda$\mathversion{normal}}}
\newcommand{\bdelta}{\text{\mathversion{bold}$\delta$\mathversion{normal}}}

\newcommand{\veps}{\varepsilon}
\newcommand{\vphi}{\varphi}
\newcommand{\bra}[1]{\{#1\}}
\newcommand{\indic}{\mathds{1}}

\begin{document}
\noindent
\begin{center}
\includegraphics[width=2.5cm]{logo}\\
\vspace{1cm}
{\huge{\bf{Balancing of Food Balance Sheets (FBSs)}}}   \\
\bigskip
{\large {\bf Marco Garieri, Natalia Golini, Luca Pozzi}} \\
{\large {\bf [name.cognome]@fao.org}} \\
{\large {\today}} \\
\end{center}




\section*{Results}

\subsection*{Assunzioni}
\begin{itemize}
\item per poter convergere i totali di riga della tabella "muTab" devo essere molto vicini ai totali di riga "veri". Quindi potremmo pensare ad un primo controllo in cui chiediamo all'algoritmo di controllare che questa assunzione sia valida prima di iniziare il running. Questo mi sembra ragionevole nella realt�. La FAO non pu� darci dei valori di riga (expected values) la cui somma si discosta troppo dalla somma dei valori veri. La definizione di "molto vicini" richiede ancora qualche simulazione.
\end{itemize}


\subsection*{Scenario 1}

Le prior usate in questo scenario hanno bounds molto stretti per tutte le colonne, ad eccezione fatta per varStock.\\


\noindent Il tempo di esecuzione per 100 iter �:

\noindent \textcolor{blue}{user  system elapsed \\
11.50    0.24  148.91}

\subsection*{Scenario 2}

In questo scenario sono state date prior con bounds pi� ampi rispetto a quelli dati per lo Scenario 1


\noindent Il tempo di esecuzione per 100 iter �:

\noindent \textcolor{blue}{user  system elapsed \\
   7.16    0.13   11.28}

\subsection*{Scenario 2}

In questo scenario sono state date prior con bounds pi� ampi rispetto a quelli dati nello Scenario 2


\noindent Il tempo di esecuzione per 100 iter �:

\noindent \textcolor{blue}{user  system elapsed \\
   4.79    0.19   10.06}



\subsection*{Appunti}

\begin{itemize}
\item per velocizzare l'algoritmo si potrebbe pensare di ordinare le colonne della tabella "muTab" in ordine crescente di sd. Rimane esclusa da questo ordinamento "varStock" che resta ad occupare l'ultima colonna della tabella.
\end{itemize}


\subsection*{Cose da fare}

\begin{itemize}
\item Individuare la tabella (se esiste o esistono) che viene pi� spesso campionata. Potrebbe rappresentare un indice di sintesi per le diverse tabelle campionate e quindi una soluzione unica da offrire a FAO.
\item Calcolare un indice di bont� di adattamento quale ad esempio il RMSE tra i valori "veri" e quelli campionati.
\item Giocare con la sd del totale di colonne. Tempo di esecuzione? Bont� di campionamento? (... ci sto pensando ...)
\end{itemize}





\end{document}